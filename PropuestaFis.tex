\documentclass[12pt]{article}

\usepackage{graphicx}
\usepackage{epstopdf}
\usepackage[english]{babel}
\usepackage[latin5]{inputenc}
\usepackage{hyperref}
\usepackage[left=3cm,top=3cm,right=3cm,nohead,nofoot]{geometry}
\usepackage{braket}
\usepackage{datenumber}

\begin{document}

\begin{center}
\Huge
Respuesta en fitness a cambios de un medio ambiente oscilante en el sistema GAL de \emph{Saccharomyces cerevisiae}.

\vspace{3mm}
\Large Juan David Estupi\~n\'an M\'endez

\large
C\'od. 201113272


\vspace{2mm}
\Large
Director: Juan Manuel Pedraza\\
Codirector: Alejandro Reyes Mu\~noz

\normalsize
\vspace{2mm}

\today
\end{center}


\normalsize
\section{Introducci�n}

La forma en que los genes se interconectan e interact\'uan entre s\'i es estudiada por el campo de la biolog\'ia de sistemas por medio de representaciones de circuitos electr\'onicos. \'Estos generalmente presentan \emph{feedbacks}, los cuales confieren ciertos estados diferenciables seg\'un c\'omo sea su configuraci\'on. Por ejemplo, se sabe que la presencia de \emph{feedbacks} positivos puede ser un indicador de un sistema multiestable \cite{Avendano}. Para el caso del sistema que degrada galactosa (GAL) de la levadura \emph{Saccharomyces cerevisiae}, en el cual existen 2 estados diferenciables y 2 estados de transici\'on, las propiedades din\'amicas de los \emph{feedbacks} les permiten a las levaduras optimizar su \emph{fitness} para un medio ambiente dado. Seg\'un resultados te\'oricos y experimentales \cite{Avendano, Murat, Kusell}, se espera que, en un medio ambiente que cambie con una frecuencia alta, el sistema GAL presente una tasa elevada de transiciones entre sus 2 estados posibles. Por el contrario, si la frecuencia de cambio es baja, la tasa de transiciones es poca \cite{Murat}. Por otro lado, la tasa de estas transiciones est\'an determinadas por la fuerza de los \emph{feedbacks} \cite{Avendano}. Esto puede ser entendido por medio de paisajes de energ\'ia \cite{Acar}, los cuales permiten relacionar los diferentes estados posibles con pozos de potencial y las transiciones con cambios estoc\'asticos determinados por la altura de la barrera de potencial. Este proyecto de grado busca entonces cuantificar c\'omo evoluciona el sistema GAL a un medio ambiente que var\'ie sus condiciones con una frecuencia determinada, analizando la fuerza de los \emph{feedbacks} y comparando el \emph{fitness} de las diferentes cepas. La importancia de estudios como estos radica en que se puede aplicar a cualquier sistema natural para explicar procesos evolutivos. Adem\'as, se pueden usar para el dise\~no y evoluci\'on dirigida de circuitos artificiales.  


\section{Objetivo General}

Cuantificar de manera experimental y por medio de simulaciones, la respuesta del sistema GAL de \emph{S. cerevisiae} a la adaptaci\'on a un medio ambiente oscilatorio. 


\section{Objetivos Espec�ficos}

\begin{itemize}
	\item Hacer experimentos de competencia entre cepas adaptadas a cierta frecuencia de cambio y cepas \emph{wild type}, para determinar el \emph{fitness} relativo entre \'estas
	\item Determinar las concentraciones de galactosa donde se presentan los diferentes estados posibles para cada una de las cepas del experimento.
	\item Aplicar el concepto de paisajes de energ\'ia para modelar las transiciones entre los posibles estados.
	\item Realizar simulaciones que permitan modelar la respuesta del \emph{fitness} en relaci\'on a la frecuencia de cambio del medio.
\end{itemize}

\section{Metodolog�a}

El proyecto cuenta con una parte experimental y otra parte computacional. La parte experimental consiste en adaptar la cepa BY4741 de \emph{S. cerevisiae} a un ambiente que cambie con una frecuencia definida. Para lograr esto se incubar\'a dicha cepa por un mes con el turbidostato del laboratorio de biof\'isica de la Universidad. Luego se realizar\'an experimentos de competencia entre la cepa adaptada y la cepa original, usando genes de fluorescencia YFP y CFP para diferencialas. Con los resultados de las competencias se determinar\'a el coeficiente efectivo de selecci\'on \cite{Hegreness}. Este proceso se realizar\'a 2 veces para obtener un resultado estad\'isticamente significativo. \\
Para la parte computacional se requiere realizar un modelo simplificado teniendo en cuenta el circuito gen\'etico, luego con los datos de los experimentos se generar\'a el modelo te\'orico que permitir\'a realizar las simulaci\'ones \cite{Mettetal}.

\section{Cronograma}
\begin{itemize}
	\item Tarea 1: Corridas de adaptaci\'on
	\item Tarea 2: Experimentos de competencias
	\item Tarea 3: An\'alisis de competencias
	\item Tarea 4: Desarrollo del modelo te\'orico
	\item Tarea 5: Simulaciones
	\item Tarea 6: Escritura del documento
\end{itemize}

\begin{table}[htb]
	\begin{tabular}{|c|cccccccccccccccc| }
	\hline
	Tareas $\backslash$ Semanas & 1 & 2 & 3 & 4 & 5 & 6 & 7 & 8 & 9 & 10 & 11 & 12 & 13 & 14 & 15 & 16  \\
	\hline
	1 & X & X & X & X &   & X & X & X & X &   &   &   &   &   &   &   \\
	2 &   &   &   &   & X &   &   &   &   & X &   &   &   &  	 &   &   \\
	3 &   &   &   &   &   & X & X &   &   &   & X & X &   &   &   &   \\
	4 & X & X &   &   &   &   &   &   &   &   & X & X & X & X &   &   \\
	5 &   &   &   &   &   &   &   &   &   &   &   &   & X & X & X &   \\
	6 &   &   &   &   &   &   &   &   &   &   &   &   &   & X & X & X \\
	\hline
	\end{tabular}
\end{table}
\vspace{1mm}


\section{Personas Conocedoras del Tema}

\begin{itemize}
	\item Juan Manuel Pedraza (Departamento de F\'isica, Universidad de los Andes, Colombia)
	\item Alejandro Reyes Mu\~noz (Departamento de Ciencias Biol\'ogicas, Universidad de los Andes, Colombia)
	\item Manu Forero Shelton (Departamento de F\'isica, Universidad de los Andes, Colombia)
	\item Daniel Cadena Ordo\~nez (Departamento de Ciencias Biol\'ogicas, Universidad de los Andes, Colombia)
\end{itemize}


\begin{thebibliography}{10}

\bibitem {Avendano} M. Avenda\~no, C. Leidy, J.M. Pedraza. Nature communications \textbf{4}:2605, (2013).

\bibitem {Murat} M. Acar, J.T. Mettetal, A. van Oudenaarden. Nature genetics \textbf{40}:4, 471-475, (2008).

\bibitem {Kusell} E. Kusell \& S.Leibler. Science \textbf{309}:5743, 2075-2078 (2005)

\bibitem {Acar} M. Acar, A. Becskei, A. van Oudenaarden. Nature \textbf{435}:7039, 228-232, (2005).

\bibitem {Hegreness} M. Hegreness, N. Shoresh, D. Hartl, R. Kishony. Science \textbf{311}:5767, 1615-1617, (2006).

\bibitem {Mettetal} J.T. Mettetal, D. Muzzey, C. G\'omez-Uribe, A. van Oudenaarden. Science \textbf{319}:5862, 482-484, (2008)

\end{thebibliography}

\section*{Firma del Director}
\vspace{1.5cm}

\section*{Firma del Codirector	}



\end{document} 